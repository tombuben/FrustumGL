\documentclass[report,11pt]{elsarticle}
%\documentclass{article}[11pt,a4]
%\documentclass[11pt]{article}

\usepackage[lined,linesnumbered]{algorithm2e}
\usepackage{a4wide}
\usepackage{ae}
\usepackage[czech]{babel}
%\usepackage[cp1250]{inputenc}
\usepackage{graphicx}
\usepackage{url}
\usepackage{pdfpages}
\usepackage{verbatim}
\usepackage{listings}
\usepackage{pgf}
\usepackage{import}

\paperwidth=210 true mm
\paperheight=297 true mm
\pdfpagewidth=210 true mm
\pdfpageheight=297 true mm

% Pro cestinu zakomentuj nasledujici radek
%\selectlanguage{english}
% Pro anglictinu zakomentuj nasledujici radek
\selectlanguage{czech}

%GET RID OF "Preprint submitted to elsevier"
\makeatletter
\def\ps@pprintTitle{%
   \let\@oddhead\@empty
   \let\@evenhead\@empty
   \def\@oddfoot{\reset@font\hfil\thepage\hfil}
   \let\@evenfoot\@oddfoot
}
\makeatother

\begin{document}

\begin{frontmatter}

\title{Semestrální práce 17\\ Hierarchical View-Frustum Culling for Z-buffer Rendering}

\author{Tomáš Bubeníček\footnote{B4M39DPG -- Tomáš Bubeníček, zimní semestr 2019/20}\\
Katedra počítačové grafiky a interakce,\\ Fakulta elektrotechnická, ČVUT Praha
}

\date{}


%%%%%%%%%%%%%%%%%%%%%%%%%%%%%%%%%%%%%%%%%%%%%%%%%%%%%%%%%%%%%%%%%%%%%%%%%%%%%%
%%
%%  Abstract
%%
%%%%%%%%%%%%%%%%%%%%%%%%%%%%%%%%%%%%%%%%%%%%%%%%%%%%%%%%%%%%%%%%%%%%%%%%%%%%%%
\begin{abstract}
Implement hierarchical view frustum culling for large scale scenes consisting of
triangles. First, construct a bounding volume hierarchy (BVH) using top-down method,
middle point subdivision. Avoid rendering such BVH nodes that cannot be visible
(out of viewing frustum) usually known as view frustum culling.
\end{abstract}

  % Klicova slova k uloze
\begin{keyword}
Frustum culling, BVH, GPU rasterization
\end{keyword}

\end{frontmatter}

%%%% --------------------------------------------------------
\section{\label{SEC:Intro}Úvod}

Při vykreslování scény pomocí metod rasterizace se pro každý vrchol provádí
transformace, jejímž cílem je projektovat body ve scéně do bodů v clip space.
V tomto souřadném prostoru lze poté snadno určit zda primitiva náležící k daným
vrcholům (v OpenGL trojúhelníky) budou ležet ve výsledném obraze. Tyto
primitiva jsou poté rasterizovány.

Pro většinu scén je při transformaci do clip-space výsledkem, že velká část
transformovaných primitiv neleží v té části clip space, ve které probíhá
rasterizace. Cílem této práce je tedy implementovat systém, který omezí počet
transformovaných vrcholů tak, aby se transformovaly pouze ty vrcholy, které
náleží k objektům, které budou po transformaci vidět na výsledném obraze.

%%%% --------------------------------------------------------
\section{\label{SEC:Description}Popis algoritmu}
%\section{\label{SEC:Description}Algorithm Description}

Algoritmus pracuje s datovou strukturou BVH (Bounding Volume Hierarchy)
postavenou nad jednotlivými objekty ve scéně. Jednotlivé uzly BVH mají osově
zarovnanou obálku ve tvaru krychle (Axis Aligned Bounding Box -- AABB) a každý
vnitřní uzel má dva potomky. Obálky jednotlivých uzlů poté testujeme vůči
pohledovému jehlanu (viewing frustum) kamery vykreslující scénu. Tento komolý
jehlan reprezentuje objem, ve kterém bude po transformaci do clip-space probíhat
samotná rasterizace.

Při testu průniku obálky a pohledového jehlanu může dojít k třem různým
výsledkům, podle kterých se při vykreslování rozhodujeme:

\begin{itemize}
    \item Obálka je mimo pohledový jehlan - Uzel a všechny jeho potomky
          nevykreslujeme.
    \item Obálka je celá uvnitř pohledového jehlanu - Uzel a všechny jeho
          potomky vykreslujeme.
    \item Obálka protíná stěnu jehlanu - Provádíme test průniku obálky na oba potomky
          uzlu, pokud uzel nemá potomky, vykreslíme.
\end{itemize}

Celý tento algoritmus je tedy založen na efektivní implementaci tohoto testu.
V následujících podkapitolách je tedy popsán algoritmus testování a několik
způsobů pro jeho zrychlení.

\subsection{Test průniku AABB-frustum}

V \cite{assarsson2000optimized} jsou popsány dva přístupy, pomocí kterých lze
test průniku obálky a pohledového jehlanu provádět. První metoda je provádět
operaci transformace do clip space na samotné vrcholy obálek, jelikož ve
výsledném souřadném systému je určit, zda obálka bude vidět na výsledném obraze
je relativně snadná - Stačí porovnat obálku transformovaného AABB a AABB
reprezentující rasterizovaný objem. Pro transformaci obálky do clip space je
třeba ale 72 operací násobení, což nemusí být ve všech implementacích efektivní
(obzvláště v implementacích nevyužívajících paralelismus či SIMD instrukcí).

Druhá, mnou implementovaná metoda, reprezentuje frustum jako šest rovin ve world
space, vůči kterým se postupně provádí testy průniku. Pokud se AABB nachází
na straně roviny reprezentující prostor vně frustum, AABB je mimo pohledový
jehlan. Naopak pokud všech šest testů průniku vrátí že se AABB nachází v
části poloprostoru reprezentujícího vnitřní část pohledového jehlanu, nachází se
obálka uvnitř. Pokud obálka protíná alespoň jednu rovinu, protíná také stěnu
výsledné obálky.

\subsection{Test průniku AABB-rovina}

Místo testování všech osmi vrcholů obálky je možné testování urychlit pomocí
testování pouze dvou bodů $n$ a $p$, ležících na úhlopříčce nejbližší normále
roviny. Určení těchto dvou bodů je snadné: Jsou určeny podle jednotlivých
komponent normálového vektoru -- Pokud je komponenta kladná, bod $p$ leží
v dimenzi komponenty na pozici bodu $max(AABB)$ a bod $n$ poté leží v dimenzi
komponenty na pozici $min(AABB)$.

Pro určení průniku je třeba rozhodnout, zda bod $n$ leží uvnitř poloprostoru
definovaného stěnou pohledového jehlanu. Pokud ne, obálka leží celá vně. Pokud
ano, provede se stejný test na bod $p$. Pokud tento bod leží vně, obálka protíná
rovinu, pokud ano, obálka leží celá uvnitř poloprostoru. Tuto implementaci bez
jakéhokoliv dodatečného zrychlení lze nalézt na obrázku \ref{fig:AABBplane},
který je založen na implementaci v \cite{sykora2002efficient}.

\begin{figure}
    \centering
\begin{lstlisting}[language=C, basicstyle=\tiny]

short AABB::checkFrustum(Frustum& f)
{
  short result = INSIDE;
  for (i = 0; i < 6; i++){
    glm::vec4& p = f.planes[i];
    m_prod = (p.x * m[p.x<0].x) + (p.y * m[p.y<0].y) + (p.z * m[p.z<0].z);
    if (m_prod > -p.w)
	    return OUTSIDE;

    n_prod = (p.x * m[p.x>0].x) + (p.y * m[p.y>0].y) + (p.z * m[p.z>0].z);
    if (n_prod > -p.w)
	    result = INTERSECTS;
  }
  return result;
}
\end{lstlisting}
    \caption{Test průniku AABB frustum pomocí šesti testů průniku s rovinou.
    Pole \texttt{this->m} má v pozici~0 minimální bod a v pozici 1 maximální bod
    osově zarovnané obálky. Pole \texttt{f.planes} obsahuje implicitní zadání
    všech šesti rovin jehlanu.}
    \label{fig:AABBplane}
\end{figure}

\subsection{Maskování rovin}

Zanořování se do vnitřních uzlů BVH nastává pouze v situaci, kdy se obálka
protíná s alespoň jednou rovinou pohledového jehlanu. Nemá ovšem smysl testovat
ve vnitřních uzlech ty roviny, o kterých již víme, že se neprotínají s rodičem.
Jedno z zrychlení algoritmu je tedy při testu průniku určit, které roviny mají
být testovány a které testovány být nemusí.



\subsection{Časová koherence}

Šance, že se mezi dvěma snímky změní rovina, u které zjistíme že se obálka
nachází vně pohledový jehlan , je relativně nízká. Jelikož ukončujeme test
průniku v moment, kdy je obálka vně jednu rovinu, je vhodné testovat rovinu která
selhala test průniku v minulém snímku jako první.

%%%% --------------------------------------------------------
\section{\label{SEC:Pitfalls}Potíže při implementaci}
%\section{\label{SEC:Pitfalls}Implementation details}

Implementaci jsem se rozhodl provést pomocí knihoven SDL a OpenGL. Nejprve jsem
se rozhodl implementaci testovat na velkých trojúhelníkových sítí poskytnutých
pro testování, což se ale při samotném vývoji ukázalo jako relativně nevhodná
volba - Tyto testovací soubory zabírají často skoro až gigabajt místa, což při
spuštění programu v debug módu může trvat mému jednoduchému .obj loaderu až
minutu a půl na načtení. Pro načítání dat jsem tedy později implementoval systém, kde
se při prvním načtení .obj souboru uloží binární soubor reprezentující data.
Při dalším načítání se nejprve zkontroluje, zda soubor existuje a pokud ano,
načte se ten. Načítání i velkého binárního souboru je poté velmi rychlé.

Samotné modely poskytnuté na testování neměly vždy přiřazené normály, takže jsem
se rozhodl vrcholům ve vertex shaderu přiřazovat náhodně barvu, aby byly
jednotlivé vrcholy rozlišitelné. Za zmínku také stojí postup na získání
jednotlivých rovin pohledového jehlanu z projekční matice, k čemuž jsem využil
postup popsaný v \cite{gribb2001fast}.

Poskytnuté modely bylo možné na grafické kartě GTX 1070 vykreslovat celé
relativně rychle (s jedním modelem \textit{asianDragon.obj} byl výsledek více jak
600 fps), takže v testovaných scénách bylo nutno těchto modelů vložit více
(cca 60). Pro definici této scény je využit textový soubor, kde je možno také
určit, zda jednotlivé modely ve scéně jsou vykreslovány v celku, či zda mají být
předzpracovány a rozděleny pomocí midpoint split metody na menší modely s předem
definovaným maximálním počtem trojúhelníků.

Při implementaci BVH jsem také vytvořil jednoduché vykreslování jednotlivých
vrstev stromu, abych se ujistil, že mnou implementované midpoint split dělení
je správně. Na semestrální práci jsem pracoval cca 100 hodin, s většinou času
strávenou na jiných věcech než je samotné programování osekávání pohledovým
jehlanem (načítání obj, vykreslování, získání rovin z projekční matice, pohyb
kamery, vytváření scén, měření výsledků psaní reportu). Na samotné tvorbě BVH
a implementaci frustum culling algoritmu jsem strávil cca 20 hodin.



%%%% --------------------------------------------------------
\section{\label{SEC:Results}Naměřené výsledky}
%\section{\label{SEC:Results}Results}

Hlavní část měření probíhala na mém domácím stolním počítači, který má
následující hardwarové a softwarové parametry:
\begin{itemize}
    \item Procesor Intel Core i5-7600K 64bit@3.80GHz (4 jádra)
    \begin{itemize}
        \item L1 4$\times$32 KB
        \item L2 4$\times$256 KB
        \item L3 6 MB
    \end{itemize}
    \item Grafická karta Nvidia GTX 1070 8GB
    \item 24 GB DDR4 RAM
    \item Samsung SSD 960 Evo 250GB NVMe disk
    \item OS Windows 10 Pro verze 1903
    \item Microsoft Visual Studio Community 2017
    \item Parametry překladače:
    \begin{itemize}
        \item Intel C++ Compiler 19.0
        \item /std=c++17
        \item /O2
        \item /Oi (Intrinsic funkce)
        \item /Qipo (Mezisouborová optimalizace)
        \item /GA (Optimalizace pro Windows OS)
        \item /Qopt-matmul (Optimalizace maticového násobení)
        \item /Qparallel (Optimalizace pomocí paralelizace)
        \item /Quse-intel-optimized-headers (Hlavičkové soubory optimalizované pro intel)
    \end{itemize}
\end{itemize}

Při testování jsem měřil několik odlišných proměnných po vteřinovém intervalu:
\begin{itemize}
  \item \textit{FPS} - Počet snímků od posledního zapsaného měření
  \item \textit{Frametime} - Doba, kterou trval výpočet posledního snímku (v milisekundách)
  \item \textit{Drawcalls} - Počet volání vykreslení na grafické kartě
  \item \textit{GLTimerQuery} - Skutečný čas vykreslování na GPU získaný pomocí OpenGL
      dotazu \texttt{GL\_TIME\_ELAPSED} (v milisekundách)
  \item \textit{CPUtime} - Čas od počátku procházení BVH až po konec včetně volání OpenGL
      vykreslovacích funkcí (v milisekundách)
  \item \textit{Traversaltime} - Čas trvání odděleně zavolaného procházení BVH,
      které nevolá žádné OpenGL příkazy a pouze navštíví potomky a spočítá počet
      objektů, které by byly zobrazeny (v milisekundách)
\end{itemize}

Důležité je zmínit, že samotná optimalizace testů průniku bude znatelná nejvíce
ve výsledku \textit{traversaltime}, jelikož ten reprezentuje pouze čas průchodu
stromem bez volání funkcí OpenGL, které poté pracují s trojúhelníkovou sítí
uloženou v jednotlivých listech.

Všechny testy jsem prováděl vícekrát (cca 10 běhů), každý průběh uložil do .csv
souboru a jednotlivé průběhy poté zprůměroval pomocí přiloženého Python skriptu.
Pro vykreslení grafů byly poté výsledky vyhlazeny konvolucí s hammingovým okénkem,
aby byl vidět průměrný výsledek ve větším časovém úseku. Pro některé scény jsem porovnával
implementaci s zapnutou optimalizací (pomocí maskování a s využitím koherence)
a bez optimalizací (testováním vždy všech šesti rovin). Také jsem otestoval scény
bez ořezávání pohledovým jehlanem.

\subsection{asianDragons-Split}

Pro hlavní část měření jsem vytvořil scénu \textit{asianDragons-Split}. Scéna obsahuje
60 instancí modelu \textit{asianDragon.obj} s velkými rozestupy a má varianty,
ve kterých  se model rozdělí do listů s maximální velikostí 25 tisíc, 50 tisíc,
100 tisíc, 200 tisíc, 500 tisíc, milion a dva miliony trojúhelníků. Samotný
model obsahuje 7,2 milionů trojúhelníků a byla také vytvořena varianta scény,
kde se model na menší části nedělí. Kamera scénou prolétává po Bézierově
křivce.

Výsledky měření této scény bez dodatečných optimalizací je vidět na
obrázku~\ref{measure:NoOptim:asianDragons-Split}. Můžeme si povšimnout, že pro
scénu s modely rozdělených do listů s 25 tisíci trojúhelníky až 200 tisíc
trojúhelníků je celková rychlost vykreslování přibližně stejná. Scéna s nejméně
trojúhelníků v listech je nepatrně rychlejší. U listů s 500 tisíců trojúhelníků
a více je poté pokles ve výkonu, kde vykreslování plného modelu bez dělení na
další listy může mít ve výsledku méně jak poloviční počet snímků za vteřinu
oproti ostatním scénám.

Je také možné si povšimnout, že moderní dedikované grafické karty
nejsou tolik limitovány samotným počtem volání vykreslovacích příkazů --
dvojnásobný nárůst v počtu vykreslení mezi scénou s 25 tisíc trojúhelníků v
listech a scénou s 50 tisíc způsobuje minimální změnu ve výsledném běhu na
grafické kartě. Kde tento výsledek známý již je je vidět při porovnávání běhu
\textit{GLTimerQuery} a \textit{CPUtime}. \textit{CPUtime} ukazuje, jak rychle
driver grafické karty předá běh zpět samotnému programu a grafická karta může
běžet asynchronně vůči CPU. V tomto případě je vidět, že drivery karty nepracují
nejlépe pokud se vykreslují příliš velké modely či pokud je volání vykreslování
příliš. Pro nejlepší využití driverů je v listech vhodné mít pro testovanou
konfiguraci přibližně 100 tisíc trojúhelníků.

Jako poslední se zaměříme na dobu samotného průchodu všemi uzly BVH bez volání
OpenGL vykreslovacích funkcí a můžeme si povšimnout, že i při stálém testování
pohledového jehlanu pomocí šesti rovin je čas traverzace cca jedna setina času
vykreslování a další optimalizace budou mít tedy minimální vliv na celkové trvání
snímku. Toto bylo potvrzeno i pomocí profilování s programem Intel VTune Amplifier,
který ukazoval výsledek, kde více jak 97\% času běhu aplikace byl běh samotného OpenGL
driveru (buď volání vykreslovacích příkazů, nebo čekání na jejich dokreslení).

Poté jsem provedl stejný test na této scéně se zapnutou optimalizací pomocí
maskování stěn a časové koherence s podobnými výsledky. Měření je možno vidět na
obrázku \ref{measure:AllOptim:asianDragons-Split} a je ve výsledném počtu snímků
za vteřinu velmi podobné. Je vidět lehké zrychlení v čase traverzace pro případ
největšího stromu (s listy s méně jak 25 tisíci trojůhelníků), a z pohledu
efektivního využití driverů je stále nejlepší využít BVH s počtem listů okolo
100 tisíc.

Pro tuto scénu jsem prováděl také měření s vypnutým ořezávání pohledovým jehlanem
(ale s trojúhelníky rozdělenými opět do různě velkých listů), které je možno vidět
na obrázku \ref{measure:NoCull:asianDragons-Split}. Výsledky jsou časově
konstantní (jelikož se nemění počet vykreslovacích příkazů) a pro asynchronní
vykreslování opět vede scéna s modely s max 100 tisíc trojúhelníky.

Čas dělení objektu \textit{asianDragon.obj} na různě velké listy, počet celkových
listů po vytvoření 60 instancí dělených objektů ve scéně a čas stavby výsledného
BVH nad celou scénou je vidět v tabulce \ref{table:desktop:asianDragon}. Pro další scény
byly časy velmi podobné.

\begin{table}[]
\begin{center}
\begin{tabular}{l|lll}
      & Čas dělení & Počet listů & Čas stavby \\ \hline
25k   & 4853 ms    & 26880       & 52.22 ms   \\
50k   & 4383 ms    & 13980       & 26.06 ms   \\
100k  & 3876 ms    & 6780        & 12.39 ms   \\
200k  & 3358 ms    & 3360        & 5.57 ms    \\
500k  & 2569 ms    & 1200        & 1.79 ms    \\
1000k & 1916 ms    & 540         & 0.87 ms    \\
2000k & 1465 ms    & 300         & 0.52 ms
\end{tabular}
\end{center}
\caption{Čas dělení objektu \textit{asianDragon.obj} a stavby BVH nad 60 instancemi děleného objektu.}
\label{table:desktop:asianDragon}
\end{table}

\subsection{asianDragons-Packed}

Scéna \textit{asianDragons-Split} je pro testování efektu optimalizace možná
nevhodná, protože mezi jednotlivými modely je relativně velká mezera a tak se
často stává, že stěnu pohledového jehlanu neprotíná žádný model ve scéně.
Tím pádem neprobíhá při traverzaci tolik testování průniku AABB-frustum, protože
se testuje pouze tehdy, když byl průnik rodičovské obálky. V jiných případech se
potomci buď rovnou vykreslí bez testování, nebo zahodí a nevykreslují vůbec.
Proto jsem vytvořil novou scénu \textit{asianDragons-Packed} podobnou scéně
předchozí s tím rozdílem, že modely mezi sebou mají podstatně menší mezery a při
průletu kamerou je jich více v průniku se stěnou pohledového jehlanu. Výsledky
těchto měření jsou na obrázcích~\ref{measure:NoOptim:asianDragons-Packed}(bez
optimalizace), \ref{measure:AllOptim:asianDragons-Packed}(s optimalizací) a
\ref{measure:NoCull:asianDragons-Packed}(s vypnutým ořezáváním).

Výsledek je prakticky totožný s předchozí scénou a větší zrychlení traverzace
není viditelné. Další testované scény jsem tedy dále netestoval ve variantě bez
optimalizací, protože rozdíl ve výsledcích byl na mé grafické kartě zanedbatelný.

\subsection{part\_of\_pompeii}

Jako další scéna byla vytvořena scéna \textit{part\_of\_pompeii}. Tato scéna
obsahuje 20 instancí modelu \texttt{part\_of\_pompeii.01.final.combined.obj},
který obsahuje cca 16 milionů trojúhelníků a byla opět testována pro různě
velký počet trojúhelníků v listech. Při testování (na obrázku
\ref{measure:AllOptim:part_of_pompeii}) byl framerate opět nejlepší pro nejjemnější
dělení a stejně jako v minulých případech byl pro nejlepší využití driverů a
asynchronního vykreslování ideální počet trojúhelníků v listu v rozmezí 50 až
100 tisíc.

\subsection{testování na dalším stroji}

Implementaci jsem také testoval na notebooku s integrovanou grafickou kartou a
těmito parametry:

\begin{itemize}
    \item Procesor Intel Core i5-8250U 64bit@3.40GHz (4 jádra, 8 vláken)
    \begin{itemize}
        \item L1 4$\times$32 KB
        \item L2 4$\times$256 KB
        \item L3 6 MB
    \end{itemize}
    \item Integrovaná grafická karta Intel® UHD Graphics 620
    \item 8 GB DDR4 RAM
    \item Seagate ST1000LM035 1TB hard disk
    \item OS Fedora 31, Linux 5.3.7
    \item Parametry překladače:
    \begin{itemize}
        \item GCC 9.2.1
        \item -std=gnu++17
        \item -O3
    \end{itemize}
\end{itemize}

Měření bylo provedeno na scéně \textit{asianDragons-Packed} a výsledky je možno
vidět na obrázku \ref{measure:LaptopAllOptim:asianDragons-Packed}. Na datech
je zřejmý slabší výkon grafické karty, ale poukazují také na rozdílnou implementaci
grafického driveru, jelikož \textit{CPUtime} je nejrychlejší pro scénu s nejméně
voláním vykreslovacích příkazů a nejvíce trojúhelníky v listech. Naopak nejdéle
trvá grafickým driverům činnost při velkém množstvím vykreslovacích příkazů --
Graf byl oříznut, aby byly vidět menší hodnoty, ale scéna s 25 tisíci trojúhelníků
v listech měla \textit{CPUtime} více jak 330 milisekund za snímek při cca 10,000
volání vykreslování.

Ačkoliv je stroj grafickým výkonem podstatně horší, rychlost dělení objektu a
stavby stromu je podstatně vyšší -- částečně způsobeno také jinou volbou překladače
a agresivnějším nastavením optimalizace. Tabulka \ref{table:laptop:asianDragon}
znázorňuje rychlost dělení a stavby stromů s různě velkými listy.

\begin{table}[]
\begin{center}
\begin{tabular}{l|lll}
      & Čas dělení & Počet listů & Čas stavby \\ \hline
25k   & 1768 ms    & 26880       & 21.18 ms   \\
50k   & 1697 ms    & 13980       & 9.67 ms   \\
100k  & 1449 ms    & 6780        & 4.04 ms   \\
200k  & 1279 ms    & 6780        & 1.89 ms   \\
500k  & 1026 ms    & 1200        & 0.67 ms    \\
1000k & 800 ms    & 540         & 0.27 ms    \\
2000k & 640 ms    & 300         & 0.18 ms
\end{tabular}
\end{center}
\caption{Čas dělení objektu \textit{asianDragon.obj} a stavby BVH nad 60 instancemi
děleného objektu na druhém stroji.}
\label{table:laptop:asianDragon}
\end{table}


%%%% --------------------------------------------------------
\section{\label{SEC:Conclusion}Závěr}

V této práci jsem implementoval algoritmus osekávání pohledovým jehlanem pomocí
hierarchické datové struktury BVH. Ačkoliv zrychlení pomocí samotného testu
intersekce je značné (závislé na pohledu kamery do scény), nepodařilo se mi
získat znatelné zrychlení při využití maskování rovin a časové koherence. Zjistil
jsem také, že pro nejlepší využití driveru grafické karty může být počet
trojúhelníků v listech závislý na zařízení a pro různé platformy je ideální
počet jiný.
%******************************************************************
% Tady nasleduje seznam literatury a to bud s pouzitim
% rucne usporadaneho seznamu ci s pouzitim system bibtex (prikaz
% \bibliographystyle{alpha} a \bibliography{dpgreport.bib}



\begin{figure}
    \begin{center}
        \input{Measurements/NoOptimization/asianDragons-Split/page.pgf}
    \end{center}
    \caption{Měření na scéně \textit{asianDragons-Split} bez dodatečných optimalizací
             s různě velkým množstvím trojúhelníků v listech. Osa $x$ reprezentuje
             čas měření v sekundách.}
    \label{measure:NoOptim:asianDragons-Split}
\end{figure}

\begin{figure}
    \begin{center}
        \input{Measurements/AllEnabled/asianDragons-Split/page.pgf}
    \end{center}
    \caption{Měření na scéně \textit{asianDragons-Split} s dodatečnými optimalizacemi
             s různě velkým množstvím trojúhelníků v listech. Osa $x$ reprezentuje
             čas měření v sekundách.}
    \label{measure:AllOptim:asianDragons-Split}
\end{figure}

\begin{figure}
    \begin{center}
        \input{Measurements/NoCulling/asianDragons-Split/page.pgf}
    \end{center}
    \caption{Měření na scéně \textit{asianDragons-Split} s vypnutým ořezáváním pohledovým
             jehlanem. Osa $x$ reprezentuje čas měření v sekundách.}
    \label{measure:NoCull:asianDragons-Split}
\end{figure}


\begin{figure}
    \begin{center}
        \input{Measurements/NoOptimization/asianDragons-Packed/page.pgf}
    \end{center}
    \caption{Měření na scéně \textit{asianDragons-Packed} bez dodatečných optimalizací.
             Osa $x$ reprezentuje čas měření v sekundách.}
    \label{measure:NoOptim:asianDragons-Packed}
\end{figure}

\begin{figure}
    \begin{center}
        \input{Measurements/AllEnabled/asianDragons-Packed/page.pgf}
    \end{center}
    \caption{Měření na scéně \textit{asianDragons-Packed} s dodatečnými
             optimalizacemi. Osa $x$ reprezentuje čas měření v sekundách.}
    \label{measure:AllOptim:asianDragons-Packed}
\end{figure}

\begin{figure}
    \begin{center}
        \input{Measurements/NoCulling/asianDragons-Packed/page.pgf}
    \end{center}
    \caption{Měření na scéně \textit{asianDragons-Split} s vypnutým ořezávání pohledovým
             jehlanem. Osa $x$ reprezentuje čas měření v sekundách.}
    \label{measure:NoCull:asianDragons-Packed}
\end{figure}


\begin{figure}
    \begin{center}
        \input{Measurements/AllEnabled/part_of_pompeii/page.pgf}
    \end{center}
    \caption{Měření na scéně \textit{part\_of\_pompeii} s dodatečnými optimalizacemi
             s různě velkým množstvím trojúhelníků v listech. Osa $x$ reprezentuje
             čas měření v sekundách.}
    \label{measure:AllOptim:part_of_pompeii}
\end{figure}


\begin{figure}
    \begin{center}
        \input{Measurements/Laptop-AllEnabled/asianDragons-Packed/page.pgf}
    \end{center}
    \caption{Měření na scéně \textit{asianDragons-Packed} na zařízení s integrovanou
             Intel GPU pod OS Linux. Osa $x$ reprezentuje čas měření v sekundách.}
    \label{measure:LaptopAllOptim:asianDragons-Packed}
\end{figure}

\bibliographystyle{abbrv}
\bibliographystyle{alpha}
\bibliography{dpgreport.bib}

\end{document}
